%!TEX root = ../main.tex
\section{Introduction}
\label{sec:Introduction}

In recent years, Cloud Computing~\cite{Armbrust:2010ee} emerged as a popular
approach to host large-scale distributed applications, as op-
posed to purchasing, provisioning and maintaining a dedicated
infrastructure. In theory, utility-driven, on-demand nature of cloud
offerings allows customers to easily and quickly provision
the exact type and amount of resources needed for a given
task. While cloud enables new possibilities for application design
and management, it also introduces a plethora of new practical 
challenges and adds a layer of complexity for system design, 
development and management.

%SLO/SLA - resource capacity guarantees vs performance guarantee 
Cloud's service consumption model inherently requires a
formal agreement between a customer (service consumer) 
and a cloud service provider. This is usually described as
Service Level Agreement (SLA)~\cite{TODO}. Key components of any
SLA are Service Level Objectives (SLOs)~\cite{TODO}. They define
specific and measurable capacity guarantees to a customer, e.g., available 
memory to a provisioned VM. 
However, customers are usually more interested in performance guarantees, whose
impact can be reflected in business key performance indicators (KPIs). 
Unfortunately, currently there is only limited support to clearly mapping the 
resource capacity requirements to the guarantees 
about workload performance.
Defining application
SLOs is largely performed ad hoc, with limited support from cloud providers. 
This poses a significant challenge to customers as it is
usually very difficult to correctly derive low-level 
resource capacity requirements, such as memory allocation,
from workload's business requirements.
%as well as to maintain them during application's life cycle.


Elasticity, one of the fundamental properties of cloud computing, allows for
applications to respond to varying load patterns by adjusting the amount of
provisioned resources to exactly match their current need, thus minimizing
over-provisioning and reducing hosting costs. 
Although elasticity offers theoretically ideal strategies to keep 
track of and enforce application's resource-bound SLOs, in practice it is 
increasingly complex to define correct elastic strategies in face of
ever-changing cloud services, novel and heterogeneous resource types (e.g., containers),
execution paradigms (e.g., serverless computing) and deployments
topologies (e.g., microservice mashes). 
In addition to resource
elasticity, the cloud computing paradigm enables additional elasticity dimensions, such
as cost elasticity and quality elasticity~\cite{Dustdar:2011dk}. 
However, current approaches dealing with SLOs largely fail to 
consider elasticity dimensions holistically. This introduces another
level of indirection for the users, who now need to consider not only 
implications of low level resources on, for example costs, but also
implications of costs on business KPIs.  

%Cloud provider's perspective. Resource underutilization
SLOs also play a crucial role for cloud service providers, as
their main business model depends on achieving best utilization
of their resource pools, while maintaining minimal level of SLA 
violations. Although cloud service providers employ numerous
techniques to optimize resource utilization in their data centers,
such as resource overcommitment and over subscription models~\cite{TODO}, 
resource bursting~\cite{noonan2016managing} and resources reclamations~\cite{engle2015capacity},
the average utilization of data centers' resource pools is 
reported to be as low as 20-30\%~\cite{TODO}. This has a negative impact 
on both the cloud service providers and the customers, because
customers are usually over provisioning for their workloads, hence 
incurring higher operational costs, while cloud service providers
are left with de facto unused resources, hence are not able to 
fully utilize their economies of scale.

% AWS Savings plan to have money commitment
% Different types of ec2 instances and multiple consumption models RI, on demand ...

In this paper we introduce SLOC -- a novel elasticity framework that promotes SLOs to 
the first-class cloud citizens. SLOC aims to raise the level of 
abstraction and provide suitable models, algorithms, runtime mechanisms and tools
for providing and consuming cloud resources in an SLO-native manner, while
at the same time offering performance guarantees to the users. 
%
Concretely, SLOC framework intends to relieve users from explicitly 
dealing with low-level concepts such as CPU and memory, by enabling 
{\em performance-driven, SLO-native approach} to cloud computing. 
Conversely SLOC aims to  
enable cloud providers to have more versatile understanding, greater control 
and better utilization of their resource pools by, among other things,
accounting for additional elasticity dimensions such as cost and quality. 

The remainder of this paper is structured as follows. 
In Section~\ref{sec:Background}, we provide an overview of our previous
work in the field of elastic computing, SLAs and cloud engineering.
In Section~\ref{sec:Challenges}, we analyze key research challenges,
which motivate our SLOC framework.
Section~\ref{sec:Approach}, presents the main vision and introduces 
the general approach of SLOC framework.
In Section~\ref{sec:mechanisms}, we discuss main techniques,  
mechanisms and the road map for our SLOC framework.
Related research is discussed in Section~\ref{sec:related_work}.
Finally, we conclude the work in Section~\ref{sec:Conclusion} and provide an outlook for
ongoing and future work.